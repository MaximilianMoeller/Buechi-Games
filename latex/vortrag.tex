% !TEX TS-program = lualatex

\documentclass{beamer}

% !TEX TS-program = lualatex

% Preamble für Uni-Mitschriften
% Stark inspiriert von https://github.com/gillescastel/university-setup/blob/master/preamble.tex
%  ┌────────┐
%  │ Basics │
%  └────────┘
%
\usepackage[TU]{fontenc} %to enable bold small-caps letters

\usepackage{fontspec}
\setsansfont{Latin Modern Roman}
% spezielle Sonderzeichen
\usepackage{textcomp}
% deutsches Sprachpaket
\usepackage[english]{babel}
%\usepackage{url}
% Einbindung von Bildern
\usepackage{graphicx}
% verbesserte floating-Objekte (z.B. figures/tables)
\usepackage{float}
% kann erkennen, ob ein space gebraucht wird oder nicht (praktisch für newcommands)
\usepackage{xspace}
% verbesserte Tabellen
\usepackage{booktabs}
% Kontrolle über enumerate, itemize und description
\usepackage{enumitem}
% Platz zwischen Paragraphen
\usepackage{parskip}
% Keine Zeilennummern auf leeren Seiten
\usepackage{emptypage}
\usepackage{subcaption}
% Text in mehreren Spalten
\usepackage{multicol}
% Lorem ipsum mit \blindtext
\usepackage[]{blindtext}
% Ersetzt etwas durch nichts (debugging)
\newcommand\hide[1]{}
% Zitate
\usepackage{csquotes}
% Bibliography
\usepackage[backend=biber, style=alphabetic]{biblatex}
%\addbibresource{references.bib}
% ifthenelse command
\usepackage{ifthen}

% e.g. and i.e.
\usepackage{xpunctuate}
\newcommand\eg{\textit{e}.\textit{g}.\xcomma}
\newcommand\ie{\textit{i}.\textit{e}.\xcomma}

% ┌────────────┐
% │ Mathematik │
% └────────────┘
% mathematische Pakete
\usepackage{amsmath, amsfonts, mathtools, amsthm, amssymb, mathrsfs}
% Terme streichen (optional mit Wert, z.B. Wert -> 0)
\usepackage{cancel}
% Bold math
\usepackage{bm, bbm}
% Shortcuts für Zahlenbereiche
\newcommand\N{\ensuremath{\mathbb{N}}}
\newcommand\Z{\ensuremath{\mathbb{Z}}}
\newcommand\Q{\ensuremath{\mathbb{Q}}}
\newcommand\R{\ensuremath{\mathbb{R}}}
\newcommand\C{\ensuremath{\mathbb{C}}}
% Funktionen und variablen aus mehreren Buchstaben
\newcommand{\var}[1]{\mathit{\text{#1}}}
\newcommand{\func}[1]{\operatorname{\text{#1}}}

% Mehr Sonderzeichen (z.B. Blitz oder eckige Doppelklammern)
\usepackage{stmaryrd}
% Fixt komische Schriftartfehler für stmaryrd
\SetSymbolFont{stmry}{bold}{U}{stmry}{m}{n}

% Widerspruch als Blitz
\newcommand\contra{\scalebox{1.5}{$\lightning$}}
% = mit "def" darüber
\newcommand\defeq{\stackrel{\text{def}}{=}}
% :<=>
\usepackage{colonequals}
\newcommand\logeq{\ratio\Longleftrightarrow}
% logical proof (inference rules)
\usepackage[]{proof}

% SI-Einheiten
\usepackage{siunitx}  
%\sisetup{locale = DE}


% ┌────────────┐
% │ algorithms │
% └────────────┘
\usepackage{algorithm}
\usepackage[noend]{algpseudocode}

%  ┌───────────┐
%  │ sonstiges │
%  └───────────┘
% some packages dont work with the beamer class
\makeatletter
\@ifclassloaded{beamer}
{
	\usepackage{xcolor}
	% TU Dresden Farben
	\usepackage{tudscrcolor}
}
{
	% Bilder auf der Titelseite
	\usepackage{titlepic}
	% verbesserter Schriftsatz (nicht sichtbare Abweichungen)
	\usepackage[]{microtype}
	% Mehr Farben
	\usepackage[dvipsnames]{xcolor}
	% TU Dresden Farben
	\usepackage{tudscrcolor}
	% Todos
	\usepackage[colorinlistoftodos]{todonotes}
	% ┌──────────────┐
	% │ Environments │
	% └──────────────┘
	\theoremstyle{definition}
	\newtheorem{definition}{Definition}[section]
	\newtheorem{theorem}{Theorem}[section]
	\newtheorem{corollary}{Corollary}[theorem]
	
	\theoremstyle{remark}
	\newtheorem*{note}{Note}
	% Referenzen
	\usepackage{hyperref}
	\usepackage[capitalise]{cleveref}
}
\makeatother


\usetikzlibrary{arrows.meta, automata, graphs, positioning, backgrounds, calc}

% use of \tikzstyle is disencouraged, see
% https://tex.stackexchange.com/questions/52372/should-tikzset-or-tikzstyle-be-used-to-define-tikz-styles
\tikzset{vertex/.style={semithick, draw, black, fill=white, inner sep=0pt, minimum size=4.5mm}}
\tikzset{eve/.style={vertex, circle}}
\tikzset{adam/.style={vertex, rectangle}}

\tikzset{my arrows/.style={->,shorten >=0.5pt,>={Stealth[round]},semithick}}
\tikzset{win/.style={color=cdblue, my arrows}}
\tikzset{other/.style={color=cdgrey, my arrows}}

\tikzset{highlight/.style={dashed, fill=#1!10, draw=#1!80},
		 highlight/.default=cdblue}

\newcommand{\convexpath}[2]{
  [   
  create hullcoords/.code={
    \global\edef\namelist{#1}
    \foreach [count=\counter] \nodename in \namelist {
      \global\edef\numberofnodes{\counter}
      \coordinate (hullcoord\counter) at (\nodename);
    }
    \coordinate (hullcoord0) at (hullcoord\numberofnodes);
    \pgfmathtruncatemacro\lastnumber{\numberofnodes+1}
    \coordinate (hullcoord\lastnumber) at (hullcoord1);
  },
  create hullcoords
  ]
  ($(hullcoord1)!#2!-90:(hullcoord0)$)
  \foreach [
  evaluate=\currentnode as \previousnode using \currentnode-1,
  evaluate=\currentnode as \nextnode using \currentnode+1
  ] \currentnode in {1,...,\numberofnodes} {
    let \p1 = ($(hullcoord\currentnode) - (hullcoord\previousnode)$),
    \n1 = {atan2(\y1,\x1) + 90},
    \p2 = ($(hullcoord\nextnode) - (hullcoord\currentnode)$),
    \n2 = {atan2(\y2,\x2) + 90},
    \n{delta} = {Mod(\n2-\n1,360) - 360}
    in 
    {arc [start angle=\n1, delta angle=\n{delta}, radius=#2]}
    -- ($(hullcoord\nextnode)!#2!-90:(hullcoord\currentnode)$) 
  }
}


% Beamer styling
\usetheme{Dresden}
\setbeamertemplate{itemize items}[circle]
\setbeameroption{hide notes} % Only slides
%\setbeameroption{show notes on second screen=bottom} % notes and slides
\AtBeginSection[]{
  \begin{frame}
  \vfill
  \centering
  \begin{beamercolorbox}[sep=8pt,center,shadow=true,rounded=true]{title}
    \usebeamerfont{title}\insertsectionhead\par%
  \end{beamercolorbox}
  \vfill
  \end{frame}
}
\renewcommand{\arraystretch}{1.3}

\author{Maximilian Moeller}
\title{Solving Büchi-Games}
\subtitle{Hauptseminar}
\date{\today}

\begin{document}

\frame{\titlepage}

\begin{frame}
\frametitle{Agenda}
\tableofcontents
\end{frame}

\section{Introduction}

\section{Algorithm I}
\begin{frame}{A Game}
	\foreach \c in {1,...,25}{
		\only<\c>{\c}
	}
	\centering
	% "{!}" = keep aspect ratios
	\resizebox{\linewidth}{!}{
		\begin{tikzpicture}[remember picture]
			\node[eve]  (0) at (0, 0) {$v_{0}$};
			\onslide<1-13,20->{
				\node[adam] (1) at (2, 0) {$v_{1}$};
				\node[eve]  (2) at (4, 0) {$v_{2}$};
				\node[adam]  (3) at (6, 0) {$v_{3}$};
			}

			\node[adam] (4) at (0, -1.5) {$v_{4}$};
			\node[eve]  (5) at (2, -1.5) {$v_{5}$};
			\onslide<1-13,20->{
				\node[adam] (6) at (4, -1.5) {$v_{6}$};
				\node[eve]  (7) at (6, -1.5) {$v_{7}$};
			}

			\onslide<1-13,20->{
				\draw (0) edge[other] (1);
			}
			\invisible<15>{\draw (0) edge [win, bend left=20] (4);}

			\onslide<1-13,20->{
				\draw (1) edge[other] (2);
				\draw (1) edge[other] (4);
			
				\draw (2) edge [other, bend left=20] (3);
				\draw (2) edge[other] (6);

				\draw (3) edge [other, bend left=20] (2);
				\draw (3) edge[other] (7);
			}
			
			\draw (4) edge [other, bend left=20] (0);
			\draw (4) edge[other] (5);
			
			\invisible<15>{\draw (5) edge [win, out=60, in=120, loop] ();}
			\onslide<1-13,20->{
				\draw (5) edge [other, bend left=20] (6);
			}

			\onslide<1-13,20->{
				\draw (6) edge [win, bend left=20] (5);
				\draw (6) edge [win] (7);

				\draw (7) edge [other, out=330, in=30, loop] ();
			}
	
		\begin{pgfonlayer}{background}
			%Backgrounds
			\onslide<2-4>{
				\only<3-4>{\draw[highlight=cdgrey!20!20, solid] \convexpath{0,5,6,5}{5mm};}

				\node[circle, highlight, minimum size=8mm] at (0) {};
				\node[circle, highlight, minimum size=8mm] at (5) {};
				\node[circle, highlight, minimum size=8mm] at (6) {};
			}
			\onslide<5>{
				\draw[highlight=cdgrey!20!20, solid] \convexpath{0,5,2,6,5,4}{5mm};
			}
			\onslide<6-7>{
				\draw[highlight=cdgrey!20!20, solid] \convexpath{0,1,2,6,5,4}{5mm};
			}
			\onslide<8-9>{
				\only<9>{\draw[highlight=cdgrey!20!20, solid] \convexpath{3,7}{5mm};}
				\node[circle, highlight, minimum size=8mm] at (7) {};
				\node[circle, highlight, minimum size=8mm] at (3) {};
			}
			\onslide<10-13>{
				\only<10>{\draw[highlight=cdgrey!20!20, solid] \convexpath{3,7,6}{5mm};}
				\only<11>{\draw[highlight=cdgrey!20!20, solid] \convexpath{2,3,7,6}{5mm};}
				\only<12-13>{\draw[highlight=cdgrey!20!20, solid] \convexpath{1,2,3,7,6}{5mm};}
			}

			\onslide<16>{
				\draw[highlight=cdgrey!20!20, solid] \convexpath{0,5}{5mm};

				\node[circle, highlight, minimum size=8mm] at (0) {};
				\node[circle, highlight, minimum size=8mm] at (5) {};
			}

			\onslide<17->{
				\draw[highlight=cdgrey!20!20, solid] \convexpath{0,5,4}{5mm};
			}

		\end{pgfonlayer}
\end{tikzpicture}}
% overprint makes sure the same amount of space is resvered on every slide -> no ‘wiggling’ of the figure
\begin{overlayarea}{\textwidth}{2cm}
\only<2,16>{
	\only<16>{Assuming slide 14 is the correct one from now on.}
	\begin{align*}
		\var{Win}' = &\left\{ u \in V_{E} \mid \exists e = (u \to v) \in E, e \in \var{Win} \right\} \\
					 &\cup \left\{ u \in V_{A} \mid \forall e = (u \to v) \in E, e \in \var{Win} \right\}		
	\end{align*}}
\only<3-15,17>{
\[
	\only<3>{\func{Attr}_{\var{Eve}}^{0}(\var{Win})=\var{Win}'}
	\only<4-6,17>{\func{Attr}_{\var{Eve}}^{k+1}(\var{Win})=\var{Win}' \cup \func{Pre}_{\var{Eve}}\left(\func{Attr}_{\var{Eve}}^{k}(\var{Win})\right)}
	\only<7>{\func{Attr}_{\var{Eve}}(\var{Win}) \neq V}
	\only<8>{V \setminus \func{Attr}_{\var{Eve}}(Win)}
	\only<9-12>{\func{Attr}_{\var{Adam}}(V \setminus \func{Attr}_{\var{Eve}}(\var{Win}))}
	\only<13-15>{\mathscr{G}' = \mathscr{G} \setminus \func{Attr}_{\var{Adam}}(V \setminus \func{Attr}_{\var{Eve}}(\var{Win}))}
\]}
\only<18-21>{
\[
	\onslide<18->{\func{Attr}_{\var{Eve}}(\var{Win}) = V'}
	\onslide<19-21>{\stackrel{\onslide<21>{(1)}}{=} W_{\var{Eve}}(\mathscr{G}')}
	\onslide<20-21>{\stackrel{\onslide<21>{(2)}}{=} W_{\var{Eve}}(\mathscr{G})}
\]}
\only<13>{This is simply not defined!
$\mathscr{G} \setminus \func{Attr}_{Adam}(F)$ is defined for $F \subseteq E$, not for $F \subseteq V$!
And neither could it mean $\mathscr{G} \setminus X$ for some $X \subseteq V$, since this is also not defined.}
\only<14>{But I guess that what they mean is $\mathscr{G}\left[ V \setminus \func{Attr}_{\var{Adam}}(V \setminus \func{Attr}_{\var{Eve}}(\var{Win})) \right]$ (see 1.7 Traps and subgames) …}
\only<15>{… and not this (despite the emphasized sublety on page 64), because here $W_{\var{Eve}} = \emptyset$.}
\end{overlayarea}
\end{frame}

\section{Algorithm II}
\begin{frame}{A Game}
	\foreach \c in {1,...,15}{
		\only<\c>{\c}
	}
	\centering
	% "{!}" = keep aspect ratios
	\resizebox{\linewidth}{!}{
		\begin{tikzpicture}[remember picture]
			\node[eve]  (0) at (0, 0) {$v_{0}$};
			\node[adam] (1) at (2, 0) {$v_{1}$};
			\node[eve]  (2) at (4, 0) {$v_{2}$};
			\node[adam]  (3) at (6, 0) {$v_{3}$};

			\node[adam] (4) at (0, -1.5) {$v_{4}$};
			\node[eve]  (5) at (2, -1.5) {$v_{5}$};
			\node[adam] (6) at (4, -1.5) {$v_{6}$};
			\node[eve]  (7) at (6, -1.5) {$v_{7}$};

			\draw (0) edge[other] (1);
			\draw (0) edge [win, bend left=20] (4);

			\draw (1) edge[other] (2);
			\draw (1) edge[other] (4);
		
			\draw (2) edge [other, bend left=20] (3);
			\draw (2) edge[other] (6);

			\draw (3) edge [other, bend left=20] (2);
			\draw (3) edge[other] (7);
			
			\draw (4) edge [other, bend left=20] (0);
			\draw (4) edge[other] (5);
			
			\draw (5) edge [win, out=60, in=120, loop] ();
			\draw (5) edge [other, bend left=20] (6);

			\draw (6) edge [win, bend left=20] (5);
			\draw (6) edge [win] (7);

			\draw (7) edge [other, out=330, in=30, loop] ();
	
		\begin{pgfonlayer}{background}
			%Backgrounds
				%\draw[highlight=cdgrey!20!20, solid] \convexpath{0,5,6,5}{5mm};
				%\node[circle, highlight, minimum size=8mm] at (0) {};
			\onslide<2-5>{
				\draw[highlight=cdgrey!20!20, solid] \convexpath{0,1,2,3,7,6,5,4}{5mm};
			}
			\onslide<5>{
				\node[circle, highlight, minimum size=8mm] at (0) {};
				\node[circle, highlight, minimum size=8mm] at (5) {};
				\node[circle, highlight, minimum size=8mm] at (6) {};
			}
			\onslide<6-11>{
				\only<6-7>{
					\draw[highlight=cdgrey!20!20, solid] \convexpath{0,1,2,6,5,4}{5mm};
				}
				\only<8-11>{
					\draw[highlight=cdgrey!20!20, solid] \convexpath{0,5,4}{5mm};
				}

				\only<7,9>{
					\node[circle, highlight, minimum size=8mm] at (0) {};
					\node[circle, highlight, minimum size=8mm] at (5) {};
				}
			}

		\end{pgfonlayer}
\end{tikzpicture}}
% overprint makes sure the same amount of space is resvered on every slide -> no ‘wiggling’ of the figure
\begin{overlayarea}{\textwidth}{2cm}
\only<2-3,6->{\[
	\only<2>{Y_{0} = V}
	\only<3, 6-10>{Y_{k+1} = \only<6,8,10>{\textcolor{cdblue}}{\func{Attr}_{\var{Eve}}}\left(\only<7,9>{\textcolor{cdblue}}{\func{Pre}_{\var{Eve}}^{\var{Win}}}(Y_{k})\right)}
	\only<11>{Y_{k+1} = Y_{k} = W_{\var{Eve}}(\mathscr{G})}
\]}
\only<4-5>{
\begin{align*}
	\func{Pre}_{\var{Eve}}^{\var{Win}}(Y) = &\left\{ v \in V_{E} \mid \exists (v \xrightarrow{\var{Win}} v') \in E, v' \in Y \right\} \\
											&\cup \left\{ v \in V_{A} \mid \forall (v \xrightarrow{\var{Win}} v') \in E, v' \in Y \right\}
\end{align*}
}
\end{overlayarea}
\end{frame}

\end{document}
