% !TEX TS-program = lualatex

% Preamble für Uni-Mitschriften
% Stark inspiriert von https://github.com/gillescastel/university-setup/blob/master/preamble.tex
%  ┌────────┐
%  │ Basics │
%  └────────┘
%
\usepackage[TU]{fontenc} %to enable bold small-caps letters

\usepackage{fontspec}
\setsansfont{Latin Modern Roman}
% spezielle Sonderzeichen
\usepackage{textcomp}
% deutsches Sprachpaket
\usepackage[english]{babel}
%\usepackage{url}
% Einbindung von Bildern
\usepackage{graphicx}
% verbesserte floating-Objekte (z.B. figures/tables)
\usepackage{float}
% kann erkennen, ob ein space gebraucht wird oder nicht (praktisch für newcommands)
\usepackage{xspace}
% verbesserte Tabellen
\usepackage{booktabs}
% Kontrolle über enumerate, itemize und description
\usepackage{enumitem}
% Platz zwischen Paragraphen
\usepackage{parskip}
% Keine Zeilennummern auf leeren Seiten
\usepackage{emptypage}
\usepackage{subcaption}
% Text in mehreren Spalten
\usepackage{multicol}
% Lorem ipsum mit \blindtext
\usepackage[]{blindtext}
% Ersetzt etwas durch nichts (debugging)
\newcommand\hide[1]{}
% Zitate
\usepackage{csquotes}
% Bibliography
\usepackage[backend=biber, style=alphabetic]{biblatex}
%\addbibresource{references.bib}
% ifthenelse command
\usepackage{ifthen}

% e.g. and i.e.
\usepackage{xpunctuate}
\newcommand\eg{\textit{e}.\textit{g}.\xcomma}
\newcommand\ie{\textit{i}.\textit{e}.\xcomma}

% ┌────────────┐
% │ Mathematik │
% └────────────┘
% mathematische Pakete
\usepackage{amsmath, amsfonts, mathtools, amsthm, amssymb, mathrsfs}
% Terme streichen (optional mit Wert, z.B. Wert -> 0)
\usepackage{cancel}
% Bold math
\usepackage{bm, bbm}
% Shortcuts für Zahlenbereiche
\newcommand\N{\ensuremath{\mathbb{N}}}
\newcommand\Z{\ensuremath{\mathbb{Z}}}
\newcommand\Q{\ensuremath{\mathbb{Q}}}
\newcommand\R{\ensuremath{\mathbb{R}}}
\newcommand\C{\ensuremath{\mathbb{C}}}
% Funktionen und variablen aus mehreren Buchstaben
\newcommand{\var}[1]{\mathit{\text{#1}}}
\newcommand{\func}[1]{\operatorname{\text{#1}}}

% Mehr Sonderzeichen (z.B. Blitz oder eckige Doppelklammern)
\usepackage{stmaryrd}
% Fixt komische Schriftartfehler für stmaryrd
\SetSymbolFont{stmry}{bold}{U}{stmry}{m}{n}

% Widerspruch als Blitz
\newcommand\contra{\scalebox{1.5}{$\lightning$}}
% = mit "def" darüber
\newcommand\defeq{\stackrel{\text{def}}{=}}
% :<=>
\usepackage{colonequals}
\newcommand\logeq{\ratio\Longleftrightarrow}
% logical proof (inference rules)
\usepackage[]{proof}

% SI-Einheiten
\usepackage{siunitx}  
%\sisetup{locale = DE}


% ┌────────────┐
% │ algorithms │
% └────────────┘
\usepackage{algorithm}
\usepackage[noend]{algpseudocode}

%  ┌───────────┐
%  │ sonstiges │
%  └───────────┘
% some packages dont work with the beamer class
\makeatletter
\@ifclassloaded{beamer}
{
	\usepackage{xcolor}
	% TU Dresden Farben
	\usepackage{tudscrcolor}
}
{
	% Bilder auf der Titelseite
	\usepackage{titlepic}
	% verbesserter Schriftsatz (nicht sichtbare Abweichungen)
	\usepackage[]{microtype}
	% Mehr Farben
	\usepackage[dvipsnames]{xcolor}
	% TU Dresden Farben
	\usepackage{tudscrcolor}
	% Todos
	\usepackage[colorinlistoftodos]{todonotes}
	% ┌──────────────┐
	% │ Environments │
	% └──────────────┘
	\theoremstyle{definition}
	\newtheorem{definition}{Definition}[section]
	\newtheorem{theorem}{Theorem}[section]
	\newtheorem{corollary}{Corollary}[theorem]
	
	\theoremstyle{remark}
	\newtheorem*{note}{Note}
	% Referenzen
	\usepackage{hyperref}
	\usepackage[capitalise]{cleveref}
}
\makeatother

