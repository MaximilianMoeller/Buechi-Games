% !TEX TS-program = lualatex

\documentclass[a4paper, cdfont=false, cdsection=false, cdgeometry=off]{tudscrartcl}

\input{preamble.tex}
\input{tikz_helpers.tex}

\hypersetup{
    pdfauthor={Maximilian Moeller},
	pdftitle={Solving Büchi Games},
    pdfsubject={Hauptseminar},
	% Colors from TUD corparate design
    unicode=true,
    colorlinks=true,
    linkcolor=cdblue,
    citecolor=cdblue,
    filecolor=cdblue,
    urlcolor=cdblue
}

\renewcommand{\arraystretch}{1.3}

\setuptodonotes{color=cdgreen!30, tickmarkheight=0.2cm, noline}

\begin{document}

% Makes all the section headings etc. use a serif font
\addtokomafont{disposition}{\rmfamily}
\addtokomafont{title}{\rmfamily}
\addtokomafont{subtitle}{\rmfamily}
\addtokomafont{author}{\rmfamily}
\addtokomafont{date}{\rmfamily}
\addtokomafont{subject}{\rmfamily}

\faculty{Faculty of Computer Science}
\institute{Institute for Theoretical Computer Science}
\chair{Chair of Algebraic and Logical Foundations of Computer Science}
\thesis{Student Research Project “Hauptseminar”}
\author{Maximilian Moeller} 
\title{Solving Büchi Games}
\supervisor{Johannes Lehmann}
\professor{Prof. Dr. Christel Baier}
\date{\today}

\maketitle[cdfont=off]

\newpage
\tableofcontents
\listoffigures
\listoftables
\listoftodos

\newpage

\begin{figure}
\resizebox{0.95\linewidth}{!}{
\begin{tikzpicture}
	\foreach \x/\owner in {0/eve, 1/adam, 2/eve, 3/eve}{
		\node[\owner] (\x) at (\x*2, 0) {$v_{\x}$};
	}
	\foreach \x/\owner in {4/adam, 5/eve, 6/adam, 7/eve}{
		\node[\owner] (\x) at ({(\x-4)*2}, -1.5) {$v_{\x}$};
	}

	\foreach \s/\t in {0/1, 1/2, 1/4, 2/6, 3/7, 4/5, 6/7}{
		\draw (\s) edge[other] (\t);
	}
	\foreach \s/\t in {4/0, 2/3, 3/2, 5/6, 6/5}{
		\draw (\s) edge [other, bend left=20] (\t);
	}

	\draw (0) edge [win, bend left=20] (4);
	\draw (5) edge [win, out=60, in=120, loop] ();
	\draw (7) edge [other, out=330, in=30, loop] ();

	\begin{pgfonlayer}{background}
		\draw[highlight] \convexpath{0,5,4}{4mm};
		\node[circle, highlight, minimum size=8.5mm] at (6) {};
	\end{pgfonlayer}
\end{tikzpicture}}
\caption[short]{long}
\label{fig:example_game}
\end{figure}

\section{Introduction}
\section{1st Algorithm}

	
\end{document}
