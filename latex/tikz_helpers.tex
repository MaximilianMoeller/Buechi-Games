\usetikzlibrary{arrows.meta, automata, graphs, positioning, backgrounds, calc}

% use of \tikzstyle is disencouraged, see
% https://tex.stackexchange.com/questions/52372/should-tikzset-or-tikzstyle-be-used-to-define-tikz-styles
\tikzset{vertex/.style={semithick, draw, black, fill=white, inner sep=0pt, minimum size=4.5mm}}
\tikzset{eve/.style={vertex, circle}}
\tikzset{adam/.style={vertex, rectangle}}

\tikzset{my arrows/.style={->,shorten >=0.5pt,>={Stealth[round]},semithick}}
\tikzset{win/.style={color=cdblue, my arrows}}
\tikzset{other/.style={color=cdgrey, my arrows}}

\tikzset{highlight/.style={dashed, fill=#1!10, draw=#1!80},
		 highlight/.default=cdblue}

\newcommand{\convexpath}[2]{
  [   
  create hullcoords/.code={
    \global\edef\namelist{#1}
    \foreach [count=\counter] \nodename in \namelist {
      \global\edef\numberofnodes{\counter}
      \coordinate (hullcoord\counter) at (\nodename);
    }
    \coordinate (hullcoord0) at (hullcoord\numberofnodes);
    \pgfmathtruncatemacro\lastnumber{\numberofnodes+1}
    \coordinate (hullcoord\lastnumber) at (hullcoord1);
  },
  create hullcoords
  ]
  ($(hullcoord1)!#2!-90:(hullcoord0)$)
  \foreach [
  evaluate=\currentnode as \previousnode using \currentnode-1,
  evaluate=\currentnode as \nextnode using \currentnode+1
  ] \currentnode in {1,...,\numberofnodes} {
    let \p1 = ($(hullcoord\currentnode) - (hullcoord\previousnode)$),
    \n1 = {atan2(\y1,\x1) + 90},
    \p2 = ($(hullcoord\nextnode) - (hullcoord\currentnode)$),
    \n2 = {atan2(\y2,\x2) + 90},
    \n{delta} = {Mod(\n2-\n1,360) - 360}
    in 
    {arc [start angle=\n1, delta angle=\n{delta}, radius=#2]}
    -- ($(hullcoord\nextnode)!#2!-90:(hullcoord\currentnode)$) 
  }
}
